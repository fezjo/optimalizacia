% % % % % % % % % % % % % % % % % % % % % % % % % % % % % % % % % % % % % % % %
\section{Deterministické zaokrúhľovanie}

V predchádzajúcej kapitole sme pri probléme \minvcover použili nasledovnú všeobecnú schému
na navrhovanie $r$-aproximačných algoritmov:

\begin{myfig}{0.7\textwidth}{svg/generalminrelax}
\end{myfig}

\noindent
Ak hľadáme minimum celočíselného programu s hodnotou $m^\star$, nájdeme (potenciálne menšie) 
minimum relaxovaného programu s hodnotou $m^\star_Q$, zaokrúhlime ho a dostaneme riešenie s hodnotou $m$.
Ak ukážeme, že pri zaokrúhľovaní hodnota riešenia stúpla najviac $r$-krát, máme zaručený $r$-aproximačný algoritmus.
To, čo sme nazvali zaokrúhlenie, však nemusí byť jednoduché aritmetické zaokrúhlenie, ale hocijaký deterministický
algoritmus. Uvedieme teraz príklad takéhoto rafinovaného zaokrúhľovania.


\subsection*{\minmulticut}

Ako motivačný príklad zoberme nasledovný problém\footnote{porov.
\cite{BBC04,EF03}}: máme databázu textov, v ktorých sa vyskytujú odkazy na
rôzne osoby a našim cieľom je rozlíšiť, kedy sa hovorí o tej istej osobe a kedy
nie. To nemusí byť také ľahké, ako sa na prvý pohľad zdá, lebo tá istá osoba sa
niekedy referencuje celým menom, niekedy iniciálami, niekedy prezývkou, niekedy
nepriamo pozíciou (napr. ''britský premiér'') a pod.  Predpokladajme, že máme k
dispozícii (napríklad ako výstup algoritmov strojového učenia na tréningovej množine dát)  
nejaký prediktor, čo je funkcia, ktorá pre dve referencie vráti
reálne číslo, pričom čím väčšia kladná hodnota, tým väčšia šanca, že ide o tú
istú osobu a čím menšia záporná hodnota, tým skôr ide o dve rôzne osoby (0
znamená, že ani prediktor netuší). Máme teda daný graf, ktorého vrcholy sú jednotlivé výskyty a hrany sú
ohodnotené kladnými alebo zápornými číslami.
Našim cieľom je rozdeliť vrcholy na komponenty, pričom každý komponent zodpovedá jednej osobe. 
Keďže prediktor nie je úplne spoľahlivý, nie vždy existuje rozdelenie, ktoré je s ním konzistentné. Chceme
preto nájsť rozdelenie, ktoré minimalizuje celkovú chybu: ak kladná hrana spája dva komponenty, alebo záporná
hrana je vovnútri komponentu, ich váha (v absolútnej hodnote) sa priráta k celkovej chybe.

\begin{myfig}{0.3\textwidth}{svg/clustering}
  Graf a jeho rozdelenie  s celkovou chybou 8: kladné hrany s váhami 4 a 1 idú medzi rôznymi komponentami a 
  záporná hrana s váhou -3 je vnútri komponentu.
\end{myfig}

\noindent
Úloha nájsť rozdelenie s minimálnou chybou je \NP-ťažká, preto sa skromne uspokojíme s približným riešením.
Pretransformujme si náš graf takto: kladné hrany zachováme. Pre každú zápornú hranu $(u,v)$ s váhou $-w$
pridáme do grafu nový vrchol $\langle u,v\rangle$, ktorý spojíme hranou váhy $w$ s vrcholom $v$.
Dostaneme tak graf s kladným ohodnotením hrán a v ňom budeme riešiť nasledovnú úlohu: chceme z neho odobrať
hrany s minimálnou váhou tak, aby v zostávajúcom grafe neboli vrcholy $\langle u,v\rangle$ a $u$ v rovnakom 
komponente súvislosti.

\begin{prob}
  Dokážte, že táto transformácia je ekvivalentná, t.j. ak máme riešenie pôvodného problému s celkovou chybou $c$,
  existuje riešenie transformovaného problému s váhou  $c$ a naopak, ak máme riešenie transformovaného
  problému s váhou $c$, tak existuje riešenie pôvodného problému s celkovou chybou $c$.
\end{prob}

\begin{myfig}{0.7\textwidth}{svg/clustering2}
  Pôvodný a transformovaný graf (keďže pôvodný graf je neorientovaný, transformácia nie je jednoznačná).
  Pridané vrcholy sú zelené, bodkovanou čiarou sú spojené vrcholy, ktoré musia byť v rôznych komponentoch.
\end{myfig}

\noindent
Transformovaný problém je špeciálnym prípadom úlohy \minmulticut:

\begin{framed}
  \begin{dfn}
    \label{dfn:multicut}
    Majme daný jednoduchý graf $G=(V,E)$ s hranami ohodnotenými nezápornými váhami, t.j. funkciou 
$\omega:E\mapsto \R^+$ a v ňom $k$ dvojíc vrcholov $(s_i,t_i)$, $i=1,\ldots,k$. Cieľom problému
\minmulticut je odobrať z grafu $G$ množinu hrán s minimálnou celkovou váhou tak, aby 
žiadna dvojica $(s_i,t_i)$ nebola v rovnakom komponente súvislosti výsledného grafu.
  \end{dfn}
\end{framed}

\begin{myfig}{0.55\textwidth}{svg/multicut}
Riešenie inštancie problému \minmulticut s celkovou cenou 12.
\end{myfig}

\noindent
Špeciálnym prípadom pre $k=1$ je problém nájdenia minimálneho rezu, ktorý je polynomiálne riešiteľný,
ale pre všeobecné $k$ je \minmulticut \NP-ťažký. Pokúsme sa teraz formulovať problém \minmulticut
ako celočíselný lineárny program. Pre každú hranu $e$  budeme mať premennú $x_e\in\{0,1\}$, ktorá bude
určovať, či sme hranu odstránili alebo nie. Chceme odstrániť hrany s minimálnou váhou, preto
chceme minimalizovať výraz $\sum_{e\in E}x_e\omega(e)$. Obmedzeniami potrebujeme zaručiť, aby vrcholy 
$(s_i,t_i)$ boli v rôznych komponentoch. Označme si ${\cal P}_{s_i,t_i}$ všetky $s_i-t_i$ cesty, t.j. všetky cesty 
v $G$, ktoré začínajú v $s_i$ a končia v $t_i$. Potrebujeme, aby na každej z nich bola vybratá aspoň
jedna hrana. Dostávame teda program (obmedzenie $x_e\le 1$ nemusíme písať explicitne, lebo
vyplýva z minimalizácie: ak je nejaké $x_e>1$, môžme ho zmenšiť na 1, obmedzenia ostanú zachované a hodnota
riešenia klesne):


\begin{equation}
\label{LP:multicut:prog1}
\begin{array}{rrcll}
  {\rm minimalizovať}     & \multicolumn{1}{l}{ \sum\limits_{e\in E}x_e\omega_e}\\
  {\rm pri\ obmedzeniach} & \sum\limits_{e:e\in\pi}x_e&\ge&1& \;\;\;
                              \forall i\in\{1,\ldots,k\},\;\; \forall \pi\in {\cal P}_{s_i,t_i}\\
                          & x_e&\ge&0& \;\;\;\forall e\in E\\
                          & x_e&\in&\Z
\end{array}
\end{equation}


\noindent
Program (\ref{LP:multicut:prog1}) teraz relaxujeme tak, že odstránime obmedzenia $x_e\in\Z$. Dostaneme lineárny program

\begin{equation}
\label{LP:multicut:prog2}
\begin{array}{rrcll}
  {\rm minimalizovať}     & \multicolumn{1}{l}{ \sum\limits_{e\in E}x_e\omega_e}\\
  {\rm pri\ obmedzeniach} & \sum\limits_{e:e\in\pi}x_e&\ge&1& \;\;\;
                              \forall i\in\{1,\ldots,k\},\;\; \forall \pi\in {\cal P}_{s_i,t_i}\\
                          & x_e&\ge&0& \;\;\;\forall e\in E\\
\end{array}
\end{equation}

\noindent
ktorý môžme interpretovať nasledovne: 
hranám chceme priradiť dĺžky $x_e$ tak, aby každá dvojica vrcholov $(s_i,t_i)$ bola vzdialená aspoň 1.
Zároveň, ak $\omega_e$ bude prierez hrany, tak $\sum_{e\in E}x_e\omega_e$ hovorí, že chceme minimalizovať 
objem všetkých hrán (hrubé hrany chceme mať krátke).
Náš relaxovaný program má ale ešte jednu chybu: má potenciálne exponenciálne veľa obmedzení, takže
simplexovým algoritmom ho nevieme efektívne vyriešiť. Upravme ho preto takto: uvažujme všetky cesty 
$\pi\in{\cal P}_{s_i,t_i}$. Namiesto explicitnej podmienky $\sum_{e\in\pi}x_e\ge1$ pre každú cestu, priraďme každému
vrcholu $v$ potenciál $p_v^{(i)}\in\R$ tak, že $p_{t_i}^{(i)}-p_{s_i}^{(i)}\ge1$. Každej ceste 
$\pi: s_i=v_1,v_2,\ldots,v_z=t_i$
prislúcha postupnosť potenciálov vrcholov $p_{v_1}^{(i)},p_{v_2}^{(i)},\ldots,p_{v_z}^{(i)}$. Ak 
vynútime, aby dĺžka každej hrany cesty bola aspoň rozdiel potenciálov, t.j.
$x_{(v_j,v_{j+1})}\ge p_{v_{j+1}}^{(i)}-p_{v_j}^{(i)}$, budeme mať zaručené, 
že celá cesta bude mať dĺžku aspoň jedna.
Namiesto programu  (\ref{LP:multicut:prog2}) nám teda stačí riešiť polynomiálne veľký program

\begin{equation}
\label{LP:multicut:prog3}
\begin{array}{rrcll}
  {\rm minimalizovať}     & \multicolumn{1}{l}{ \sum\limits_{e\in E}x_e\omega_e}\\
  {\rm pri\ obmedzeniach} & x_e&\ge&p_v^{(i)}-p_u^{(i)}& \;\;\;
                              \forall i\in\{1,\ldots,k\},\;\; \forall e\in E, \;e=(u,v)\\
                          & x_e&\ge&p_u^{(i)}-p_v^{(i)}\\
                          & p_{t_i}^{(i)}-p_{s_i}^{(i)}&\ge&1& \;\;\;
                              \forall i\in\{1,\ldots,k\}\\
\end{array}
\end{equation}

\noindent
Na jednej strane, každé riešenie programu (\ref{LP:multicut:prog3}) spĺňa podmienky programu  (\ref{LP:multicut:prog2}),
na druhej strane, ak máme riešenie programu  (\ref{LP:multicut:prog2}), nastavíme potenciály $p_{v}^{(i)}$ ako
vzdialenosť $v$ od $s_i$, kde za dĺžku hrany berieme $x_e$. Ľahko vidno, že takto nastavené potenciály spĺňajú podmienky
programu  (\ref{LP:multicut:prog3}), a teda programy  (\ref{LP:multicut:prog2}) a  (\ref{LP:multicut:prog3}) sú ekvivalentné.


\noindent
Program (\ref{LP:multicut:prog3}) môžme vyriešiť napr. simplexovým algoritmom a dostaneme optimálne riešenie
s hodnotami premenných $x_e^\star$ a cenou $m_Q^\star=\sum_{e\in E}x_e^\star\omega_e$. Keby všetky $x_e^\star\in\{0,1\}$,
mali by sme priamo riešenie programu (\ref{LP:multicut:prog1}) a teda aj problému \minmulticut. 
Keď to skúsime s grafom z predchádzajúceho obrázka, zistíme, že optimálne riešenie  (\ref{LP:multicut:prog3})
je celočíselné. To vyzerá nádejne, tak môžme skúsiť napríklad takýto experiment: zoberieme všetkých $510 489$ kubických
(t.j každý vrchol je susedný s troma inými) grafov na $20$ vrcholoch, váhy hrán ponecháme na 1, 
a pre každý graf si náhodne vygenerujeme  $k\in\{1,\ldots,20\}$ dvojíc $s_i, t_i$. Experiment dopadol tak,
že spomedzi $510 489$ riešení bolo $268 178$ ($52.5\%$) celočíselných, $135 328$ ($26.5\%$) poloceločíselných
a dokopy bolo $458 008$ ($89.7\%$) riešení s najmenšou nenulovou hodnotou $\ge\frac{1}{4}$.
Jednoduchým zaokrúhlením všetkého nenulového nahor preto v $89.7\%$ prípadov dostaneme prinajhoršom 4-aproximáciu.
Prv, než by nás nedôslednosť zviedla k nejakým unáhleným záverom, pozrime si bližšie tých $10.3\%$
zvyšných prípadov. 
Skutočnosť totiž môže byť taká, že takéto zlé prípady nie sú až také ojedinelé, iba v našich testovacích 
dátach sa vyskytujú zriedkavo.
Zlé boli napospol prípady, v ktorých bolo veľké $k$, t.j 
veľa dvojíc $s_i,t_i$. Zoberme si niektorý z testovaných grafov a skúsme vyriešiť inštanciu,
do ktorej zahrnieme všetky dvojice vrcholov vo vzdialenosti aspoň $4$.

\begin{myfig}{\textwidth}{svg/multicut2}
  Vľavo je graf s 20 vrcholmi a 30 hranami (všetky hrany majú váhu 1). 
  Uvažujeme inštanciu \minmulticut, v ktorej treba oddeliť každú dvojicu
vrcholov vo vzdialenosti aspoň 4 (priemer grafu je 5, dvojíc vo vzdialenosti aspoň 4 je 22). V strede je optimálne
riešenie relaxovaného LP (čísla na hranách sú násobky $\frac{1}{15}$) s hodnotou 7. Keďže iba päť hrán
v riešení je nulových, zaokrúhlením nahor získame riešenie s hodnotou 25. Vpravo je celočíselné riešenie
s hodnotou 8: po odstránení 8 čiarkovaných hrán sa graf rozpadne na tri komponenty a každý z nich má priemer najviac 3. 
\end{myfig}


\noindent
Vidíme, že jednoduché zaokrúhlenie môže dávať zlé výsledky, a preto má zmysel rozmýšľať nad lepším
''zaokrúhľovacím'' algoritmom.
Označme $m^\star$ optimálnu cenu riešenia problému \minmulticut: platí
platí $m_Q^\star\le m^\star$. Ukážeme si teraz, ako ''zaokrúhliť'' hodnoty $x_e^\star$ (t.j. vybrať
hrany do rezu) tak, aby sme
dostali riešenie problému \minmulticut s cenou najviac $4\ln(2k)m_Q^\star$.

\noindent
Náš ''zaokrúhľovací'' algoritmus 
bude postupne z grafu ''vyhrýzať'' množiny $V_1,V_2,\ldots,V_k$. V prvej iterácii nájde množinu $V_1$ tak,
že $s_1\in V_1$, $t_1\not\in V_1$ a zároveň žiadna dvojica $s_i, t_i$ nie je vo $V_1$. Algoritmus vyberie
do rezu hrany oddeľujúce $V_1$ od zvyšku grafu a pokračuje s grafom na vrcholoch $V-V_1$. V $i$-tej iterácii,
ak $s_i$ alebo $t_i$ už nie sú v grafe (aspoň jeden z nich tam určite bude), tak $V_i=\emptyset$, ináč
sa nájde $V_i$ tak, aby $s_i\in V_i$ a $t_i\not\in V_i$ a aby žiadna dvojica $s_j,t_j$ nebola vo $V_i$;
$V_i$ sa odstráni z grafu (t.j. vyberú sa do rezu všetky hrany, ktoré oddeľujú $V_i$).  
Algoritmus takto postupne vytvorí rez, ktorý odseparuje každú dvojicu $s_i, t_i$. 
Ako vyberať množiny $V_i$ a využiť pri tom optimálne riešenie relaxovaného problému?


\noindent
Zoberme si hodnoty $x_e^\star$ a predstavme si optimálne riešenie (\ref{LP:multicut:prog2})  ako sieť potrubí, 
kde hrana $e$ má prierez $\omega_e$ a
dĺžku $x_e^\star$, t.j. má objem $\omega_ex_e^\star$. Pre hranu $e=(u,v)$ označme $d(u,v)=x_e^\star$ a 
prirodzeným spôsobom môžme merať vzdialenosť ľubovoľných dvoch vrcholov $d(v_1,v_2)$. 
Keď algoritmus vyberie nejakú množinu $V_i$, musí prerezať všetky potrubia, ktoré ju oddeľujú od zvyšku grafu,
pričom za prerezanie potrubia $e$ zaplatí jeho prierez, t.j. $\omega_e$.
Z prípustnosti riešenia vieme, že pre každú dvojicu $s_i,t_i$ je $d(s_i,t_i)\ge1$ a 
optimálne riešenie  (\ref{LP:multicut:prog2})  minimalizuje
celkový objem hrán, t.j. 
$$\Psi:=\sum_{e=(u,v)\in E}\omega_ed(u,v)$$
Označme $G_r'=(V_r',E_r')$ graf, ktorý vznikne po odstránení množín 
$V_1,\ldots,V_{r-1}$ z grafu $G$. Prirodzeným spôsobom si zadefinujeme pojem gule s polomerom $\rho$:
$${\cal B}_\rho(v)=\{u\in V_r'\mid d(u,v)\le\rho\}$$
Ak $G_r'$ obsahuje $s_r$ aj $t_r$, 
množina $V_r$ bude guľa vhodného polomeru $\rho$ okolo vrchola $s_r$. Ostáva nám určiť, ako vybrať $\rho$. V prvom rade
chceme, aby $\rho<1/2$, čo nám zaručí, že žiadna dvojica $s_j,t_j$ 
nebude ležať vo $V_r$ (vzdialenosť každej dvojice je aspoň 1).
Nech vnútorné hrany gule sú
$${\cal E}_\rho(v)=\{(w,z)\in E_r'\mid w,z\in{\cal B}_\rho(v)\}$$
a hranová hranica gule je
$$\overline{\cal E}_\rho(v)=\{(w,z)\in E_r'-{\cal E}_\rho(v)\mid w\in{\cal B}_\rho(v)\vee z\in{\cal B}_\rho(v)\}$$
Objem gule definujeme (mierne neštandardne) tak, že okrem objemu hrán pridáme z technických dôvodov člen $\Psi/k$:
$$V_\rho(v)=\frac{\Psi}{k}+\sum_{(w,z)\in{\cal E}_\rho(v)}\omega_{(w,z)}d(w,z)+
\sum_{(w,z)\in\overline{\cal E}_\rho(v)}\omega_{(w,z)}\left(\rho-\min(d(v,w),d(v,z))\right)$$
Vnútorné hrany gule prispievajú do jej objemu celým svojím objemom, hrany z hranovej hranice iba príslušnou časťou.
Najviac nás, pochopiteľne, zaujíma cena gule, t.j. veľkosť jej hranového rezu:

\noindent
\begin{minipage}[t]{0.5\textwidth-1cm}
\noindent
$$C_\rho(v)=\sum_{(w,z)\in\overline{\cal E}_\rho(v)}\omega_{(w,z)}$$
Konečne sa dostávame k tomu, ako zvoliť polomer $\rho$ v množine $V_r$. Chceme, aby guľa, ktorú odseparujeme mala čo 
najmenšiu jednotkovú cenu, t.j.
$$F_\rho(v)=\frac{C_\rho(v)}{V_\rho(v)}$$
Funkcia $F:\rho\mapsto F_\rho(v)$ pre daný vrchol $v$ má body nespojitosti pre tie hodnoty $\rho$, pre ktoré
existuje vrchol $w$ vo vzdialenosti $\rho$ od $v$. Na každom intervale medzi bodmi nespojitosti je $F$ diferencovateľná
a klesajúca.
Preto môžme ľahko nájsť minimum $F_\rho(s_i)$ na intervale $(0,1/2)$: vyberieme minimum z hodnôt 
$F_\rho(s_i)$ pre $\rho=\delta(s_i,u)-\varepsilon$.
Ostáva nám ukázať, že takto získaný rez je naozaj dobrou aproximáciou. Na to ukážeme nasledovné tvrdenie:
\end{minipage}\hspace*{1cm}\begin{minipage}[t]{0.5\textwidth}
\begin{myfig}{0.7\textwidth}{svg/multicut3}
  Predstava relaxovaného riešenia ako siete potrubí. Pri hranách je napísaná ich dĺžka ($x_e^\star$) a v zátvorke 
  prierez ($\omega_e$). 
  Guľa s polomerom $0.4$ má cenu $13$ a objem $\frac{\Psi}{k}+4.65$. Ak polomer vzrastie na $0.6$, cena ostane rovnaká,
  ale objem stúpne.
\end{myfig}
\end{minipage}
\noindent
\begin{equation}
\label{eq:cut1}
\forall v\exists\rho<1/2:\;F_\rho(v)\le2\ln(2k)
\end{equation}

\begin{myfig}{\textwidth}{svg/multicut4}
  Funkcia $F$ pre jeden vrchol $v$ z predchádzajúceho príkladu. Optimum je $\Psi=7$ a $k=22$.
  Do vzdialenosti $\frac{2}{15}\approx 0.13$ je vnútro gule obsahuje iba vrchol $v$ a veľkosť rezu
  je 3, preto pre $0\le\rho<\frac{2}{15}$ je $F_\rho(v)=\frac{3}{\frac{7}{22}+3\rho}$.
  Pre $\frac{2}{15}\le\rho<\frac{3}{15}=0.2$ vnútro gule obsahuje dva vrcholy a celú hranu dĺžky $\frac{2}{15}$;
  veľkosť rezu je 4, preto na tomto intervale $F_\rho(v)=\frac{4}{\frac{7}{22}+\frac{2}{15}+2\rho+2(\rho-\frac{2}{15})}
  \approx\frac{4}{0.18+4\rho}$. Ďalej až po vzdialenosť $\frac{5}{15}=\frac{1}{3}$ 
  vnútro obsahuje tri vrcholy a dve hrany s celkovým objemom $0.2$, pričom v reze je 5 hrán, atď.
\end{myfig}

\noindent
Z tvrdenia (\ref{eq:cut1}) vyplynie požadovaná aproximácia: nech algoritmus vyberie gule ${\cal B}_{\rho_1}(s_{i_1}),\ldots,
{\cal B}_{\rho_h}(s_{i_h})$. Cena rezu je
$$m=\sum_{j=1}^hC(s_{i_j},\rho_j)\le2\ln(2k)\sum_{j=1}^hV(s_{i_j},\rho_j)$$
V poslednej sume sa členy $\Psi/k$ z definície objemu zosumujú najviac do $\Psi$ a zvyšné objemy opäť
najviac do $\Psi$, a teda $m\le2\ln(2k)2\Psi$.

\vskip 4pt
\noindent
Dôkaz dokončíme dôkazom tvrdenia (\ref{eq:cut1}): Fixujme si vrchol $v$ a uvažujme všetky funkcie
ako funkcie vzdialenosti $\rho$. Ak $V(\rho)$ je diferencovateľná, platí
$V'(\rho)=C(\rho)$. Preto
$$F(\rho)=\frac{V'(\rho)}{V(\rho)}=\left[\ln V(\rho)\right]'$$
Tvrdenie (\ref{eq:cut1}) dokážeme sporom: predpokladajme, že pre všetky $\rho<1/2$ platí 
$F(\rho)>2\ln(2k)$. Ak $F$ je diferencovateľná na intervale $(0,1/2)$, máme
$$\left[\ln V(\rho)\right]'>2\ln(2k)$$
obe strany zintegrujeme určitým integrálom od $0$ po $1/2$ a dostaneme
$$\ln\frac{V\left(\frac{1}{2}\right)}{V(0)}>\ln2k$$
a odtiaľ 
$$V\left(\frac{1}{2}\right)>2kV(0)=2\Psi$$
čo je spor, lebo $V(\rho)\le\Psi+\Psi/k$.
Ak $F$ nie je diferencovateľná na intervale $(0,1/2)$, zopakujeme predchádzajúcu úvahy na každom inetrvale
spojitosti a výsledky sčítame.


\vskip 2ex
\noindent
Predchádzjúce úvahy môžme zhrnúť do tvrdenia:
\begin{veta}
  Existuje algoritmus, ktorý pre problém \minmulticut vráti v polynomiálnom čase riešenie s veľkosťou
  nanajvýš $4\ln(2k)$-násobku optima.
\end{veta}

\noindent
Garancia, ktorú sme práve dokázali, sa zdá byť pomerne slabá a čakali by sme, že vo veľa prípadoch budú skutočné
výsledky nášho algoritmu lepšie. Je to naozaj tak? Môžme prípadne dokázať silnejšiu garanciu?
Ukážeme si, že to nie je možné. Zoberme si regulárny graf $G$, ktorý má $n$
vrcholov a každý vrchol má stupeň $d\ge3$. Zvoľme si nejaký parameter $\alpha$ a uvažujme inštanciu
\minmulticut na grafe $G$, kde všetky hrany majú váhu 1 a dvojice $s_i,t_i$, ktoré treba rozpojiť,
sú všetky dvojice vrcholov vo vzdialenosti aspoň $\alpha$. Keď pre každú hranu $e$ zvolíme $x_e=\frac{1}{\alpha}$,
dostaneme prípustné riešenie programu (\ref{LP:multicut:prog2}), a preto $m^\star_Q\le\frac{|E|}{\alpha}$.
Ukážeme teraz, ako docieliť, aby ľubovoľné celočíselné riešenie 
malo veľkosť aspoň $n$. 
Nech $M$ je optimálne celočíselné riešenie, t.j. množina hrán, ktorá rozpojí všetky vrcholy
vo vzdialenosti aspoň $\alpha$. Odstránením $M$ z $G$ dostaneme graf $G'=(V,E-M)$
s niekoľkými komponentami súvislosti, pričom platí:

\begin{lema}
  Každý komponent súvislosti $G'$ má najviac $d^\alpha$ vrcholov.
\end{lema}
\begin{dokaz}
  Predpokladajme sporom, že by existoval komponent $K$ s viac ako $d^\alpha$ vrcholmi. Zoberme si hocijaký 
  vrchol $v\in K$. Vrchol $v$ má v $G$ stupeň $d$, teda aj v $K$ má najviac $d$ susedov. Každý z nich 
  má opäť najviac $d$ susedov, preto v celom $G$ (a teda aj v $K$) môže byť najviac $d+d(d-1)<d+d^2$ vrcholov
  vo vzdialenosti najviac $2$ od $v$. Pokračujme indukciou a dostaneme, že môže byť
  najviac $d+d^2+\cdots+d^{\alpha-1}$ vrcholov vo vzdialenosti nanajvýš $\alpha$ od $v$.
  Lenže $1+d+d^2+\cdots+d^{\alpha-1}=\frac{d^\alpha-1}{d-1}<d^\alpha$, preto v $K$ existuje vrchol $w$, ktorý
  je od $v$ ďalej ako $\alpha$. Vrcholy  $v,w$ ale tvorili nejakú dvojicu $s_i, t_i$, 
  preto nesmú byť v jednom komponente, čo je 
  spor.
\end{dokaz}

\noindent
Nech teraz $S_1,\ldots,S_h$ sú komponenty súvislosti $G'$. 
Označme $\delta(S)$ hranovú hranicu množiny $S$, t.j. tie hrany, ktoré majú jeden koniec v $S$ a druhý mimo $S$.
Pretože každá hrana má dva konce, platí $2|M|=\sum_{i=1}^h|\delta(S_i)|$. Pretože každý vrchol je v nejakom komponente
súvislosti (vyhadzovali sa iba hrany), máme $\sum_{i=1}^h|S_i|=n$.
V ďalších  úvahách použijeme užitočný nástroj: expandéry. Sú to grafy, v ktorých z každej množiny vrcholov
odchádza ``veľa'' hrán.

\noindent
\begin{framed}
  \begin{dfn}
    Graf $G=(V,E)$ nazveme {\em expandérom}, ak pre každú množinu vrcholov $S\subset V$ platí
    $$|\delta(S)|\ge\min \{ |S|, |\overline{S}|\},$$
    kde $\overline{S}=V-S$. 
  \end{dfn}
\end{framed}

\noindent
Predpokladajme teraz, že náš graf $G$ je expandér a zvoľme si $\alpha=\lfloor\log_d n/2\rfloor$. Pretože
$d^\alpha\le n/2$, každý komponent súvislosti v $(V,E-M)$ má najviac $n/2$ vrcholov, a teda dostávame
$$|M|=\frac{1}{2}\sum_{i=1}^h|\delta(S_i)|\ge\frac{1}{2}\sum_{i=1}^h|S_i|=\frac{n}{2}.$$
Keď si to zhrnieme, máme inštanciu na grafe stupňa $d$ s $n$ vrcholmi, kde optimálny multirez má $\Omega(n)$
hrán, ale existuje relaxované riešenie  s cenou $O\left(\log d\frac{n}{log n}\right)$. Zároveň máme
$k=\Theta(n^2)$ dvojíc $s_i,t_i$, lebo pre každý vrchol je aspoň $n/2$ vrcholov vo vzdialenosti aspoň $\alpha$.
Ak chceme docieliť, aby parametre $n$ a $k$ boli nezávislé, stačí pre nejaké $\ell$ nahradiť 
každú hranu cestou dĺžky $\ell$ a ponechať dvojice $s_i, t_i$: počet vrcholov narastie, ale aproximačný
faktor ostane rovnaký. Vidíme teda, že náš zaokrúhľovací algoritmus nemôže dať v najhoršom prípade
lepší výsledok ako $O(\log k)$-násobok optima.

\noindent
Ostáva posledná otázka, či vôbec expandéry existujú a či sú zriedkavé. Existujú a vyskytujú sa často, ale
sú plaché. Nájsť deterministické konštrukcie, ktoré by zaručene vyrobili
expandér je pomerne náročné. Na druhej strane, nie je príliš ťažké ukázať (aj keď to tu neurobíme),
že väčšina regulárnych grafov sú expandéry.

\IGNORE{ %% TODO DO APPENDIXU
Dá sa však ukázať, že náhodne vybratý regulárny graf je s veľkou pravdepodobnosťou expandér.
Na dôkaz tohto tvrdenia by sme ale potrebovali povedať, čo myslíme "náhodne vybratým" grafom --
špecifikovať pravdepodobnostné rozdelenie, z ktorého vyberáme. Intuitívne je asi očakávané,
že zoberieme všetky $n$-vrcholové $d$-regulárne grafy (neočíslované, t.j. izomorfné
grafy považujeme za jeden a ten istý) a z nich vyberieme rovnomerne náhodne. S týmto rozdelením
sa ale zle pracuje, ničmenej pre záujemcov uvedieme jednoduchšie tvrdenie, ktoré zachytáva
hlavné myšlienky dôkazu:


\begin{veta-star}
  Existuje randomizovaná procedúra, ktorá pre zadané $n$, $d$ s veľkou pravdepodobnosťou nájde 
  $n$-vrcholový $d$-regulárny expandér.
\end{veta-star}

\begin{dokaz}
  Zoberme si $d$-ticu permutácií $n$ prvkov $\pi_1,\ldots,\pi_d$; takýto objekt budeme volať {\em vrtuľa}
  a vieme z nej zostrojiť  bipartitný graf s $n$-prvkovými partíciami $A$, $B$ tak, že $i$-ty 
  vrchol z $A$ spojíme
  s vrcholmi $\pi_1(i),\pi_2(i),\ldots,\pi_d(i)$ (náš graf možno bude mať násobné hrany, ale to nevadí).
  Každá permutácia definuje úplné párovanie (1-faktor) a vzniknutý bipartitný graf je zjednotením $d$
  1-faktorov. 
  Vrtuľu nazveme {\em zlá}, ak v takto zostrojenom grafe existuje nejaká množina $X\subset A$ a 
  množina $Y\subset B$ tak, že $|X|\le\frac{n}{2}$, $|Y|\le\frac{3}{2}|X|$ a všetky hrany z $X$ končia v $Y$, t.j.
  $$\forall i\in X\;j\in\{1,\ldots,d\}:\;\pi_j(i)\in Y,$$
  resp. symetricky pre $X\subset B$ a $Y\subset A$.
  Jadrom dôkazu bude ukázať, že náhodne vybratá vrtuľa je dobrá. Z dobrej vrtule totiž vieme vyrobiť expandér tak,
  že ''zlepíme'' bipartície $A$ a $B$.
  \begin{myfig}{0.8\textwidth}{svg/vrtula}
  Vrtuľa $[2,5,1,8,3,6,4,7]$, $[3,1,5,2,3,4,7,8]$ a vzniknutý graf s 8 vrcholmi a stupňom najviac 4.
 \end{myfig}
  \noindent
  Vo výslednom grafe odstránime slučky a zrušíme násobné hrany a dostaneme graf $G$, ktorý má $n$ vrcholov a maximálny 
  stupeň $2d$. Predpokladajme, že pôvodná vrtuľa bola dobrá; zoberme si ľubovoľnú
  množinu vrcholov $S$ v grafe $G$ a nech $|S|=t$. Zodpovedajúce vrcholy v partícii $A$ sa zobrazia
  na aspoň $\frac{3}{2}t$ vrcholov v partícii $B$
  
\end{dokaz}


\nocite{L94}

%TODO: bin packing, facility location, ... -- ani nie, potrebuju elipsoidovu metodu
}

\subsection*{Iterované zaokrúhľovanie}

\noindent
V prípade \minvcover sme vyriešili relaxovaný program a podarilo sa nám ukázať, že v bázovom riešení
relaxovaného programu má každá premenná nejakú dobrú vlastnosť (konkrétne, že $x_i\in\{0,\frac{1}{2},1\}$),
na základe ktorej sme mohli odhadnúť zaokrúhľovaciu chybu. Teraz si ukážeme, čo sa dá robiť, ak dobrú
vlastnosť vieme ukázať iba o niektorých hodnotách relaxovaného riešenia. Pri použití techniky iterovaného 
zaokrúhľovania 
vyriešime relaxovaný lineárny program, potenciálne vyberieme niekoľko premenných, ktorých hodnoty zafixujeme 
a zo zvyšku vyrobíme menší lineárny program, na ktorom celý postup opakujeme. 

\noindent
Ukážeme si túto techniku na príklade.
Predstavme si typografickú firmu, ktorá má k dispozícii rôzne tlačiarne s rôznymi technológiami.
Firma dostane zákazku, ktorá pozostáva z niekoľkých úloh. Každá úloha sa potenciálne dá vytlačiť na 
rôznych tlačiarňach, s rôznym časom spracovania a nákladmi (napríklad čiernobiely text sa dá tlačiť na 
offsetovej tlačiarni, alebo laserovej tlačiarni, alebo aj na farebnom plotri -- tam to bude ale dlho trvať
a bude to drahé). Zákazka má na spracovanie časový limit a cieľom je naplánovať úlohy na tlačiarne tak, aby sa
stihol termín a zároveň náklady boli čo najmenšie. To nás vedie k nasledujúcej formulácii:

\vbox{
\begin{framed}
  \begin{dfn}
    \label{dfn:mingap}
    Majme množinu procesorov $M$ a množinu úloh $J$, kde $|M|=m$ a $|J|=n$. Pre každú dvojicu $i\in M$, $j\in J$ máme 
    daný čas $t_{ij}\ge 0$ a cenu $c_{ij}\ge 0$ spracovania úlohy $j$ na procesore $i$. 
    Zároveň je dané číslo $T$. Cieľom problému \mingap je priradiť
    každej úlohe $j\in J$ procesor $u_j\in M$ tak, aby sa minimalizovala celková cena priradenia, t.j.
    $\sum_{j\in J}c_{u_jj}$. Zároveň sa každý procesor musí zmestiť do časového limitu $T$, t.j.
    pre každé $i\in M$ platí \hbox{$\sum_{j:u_j=i} t_{ij}\le T$.}
  \end{dfn}
\end{framed}
}

\noindent
Prirodzená vizualizácia je pomocou bipartitného grafu $G=(V,E)$  s bipartíciou $V=M\uplus J$, pričom 
pre dvojicu $i\in M$, $j\in J$ je $(i,j)\in E$, ak $t_{ij}\le T$.

\begin{myfig}{0.65\textwidth}{svg/assignment}
Optimálne priradenie s cenou $12$ pre limit $T=10$.
Prvé číslo na hrane udáva cenu, druhé čas.
\end{myfig}

\noindent
Problém \mingap je algoritmicky ťažko riešiteľný. Aj keď uvažujeme veľmi špeciálny prípad dvoch procesorov
s tým, že pre každú úlohu $p_{1j}=p_{2j}=p_j$ (t.j. časy spracovania sú na obidvoch procesoroch rovnaké)
a $\sum_{j\in J}p_j=2T$, rozhodnúť, či sa dajú úlohy podeliť tak, aby na každom procesore bol čas najviac $T$
je ekvivalentné známemu \NP-úplnému problému \probname{Min-Partition}\footnote{
  Problém \probname{Min-Partition} je definovaný nasledovne: pre množinu čísel $A=\{a_1,\ldots,a_n\}$
  treba rozhodnúť, či existuje partícia $A=B\uplus C$ taká, že $\sum_{x\in A}x=\sum_{x\in B}x$.
}.
Neočakávame preto, že by sme ho vedeli efektívne riešiť. Ukážeme riešenie trochu jednoduchšej úlohy.
Povedzme, že naša typografická firma pri prevzatí zákazky sľúbila dodaciu dobu ''{\em 5 až 10 dní}''.
Budeme sa snažiť riešiť úlohu s termínom $T=5$ dní, ale ak ho trochu zmeškáme, nič zlé sa nestane, kým
to nebude viac ako 10 dní. Algoritmus, ktorý ukážeme, bude myslený na práve takúto situáciu.
Bude riešiť problém \mingap s tým, že nájde
priradenie s optimálnou cenou. Čas môže byť aj väčší ako $T$, ale nikdy neprekročí $2T$
(tu si treba povšimnúť, že to nie je to isté, ako riešiť priamo problém s limitom $2T$, lebo ten môže
mať menšiu optimálnu cenu).

\noindent
Problém \mingap si najprv formulujeme ako celočíselný lineárny program. Pre každú dvojicu
$i\in M$, $j\in J$ takú, že $(i,j)\in E$ si zavedieme premennú $x_{ij}\in\{0,1\}$, ktorá vyjadruje,
či procesor $i$ má pridelenú úlohu $j$. Definícia~\ref{dfn:mingap} nám priamočiaro dáva ILP:


\begin{equation}
\label{eq:mingap:ilp}
\begin{array}{rrcll}
  {\rm minimalizovať}     & \multicolumn{1}{l}{ \sum\limits_{(i,j)\in E}x_{ij}c_{ij}}\\[3ex]
  {\rm pri\ obmedzeniach} & \sum\limits_{i\in M} x_{ij}& = &1 & \;\;\; \forall j\in J\\[3ex]
                          & \sum\limits_{j\in J} t_{ij}x_{ij}&\le & T & \;\;\; \forall i\in M\\[3ex]
                          & \multicolumn{3}{c}{ x_{ij}\in\{0,1\}} & \;\;\; \forall i\in M, \; \forall j\in J
                        %  & x_{ij} & \ge & 0 & \;\;\; \forall i\in M, \; \forall j\in J
\end{array}
\end{equation}

\noindent
Relaxácia programu (\ref{eq:mingap:ilp}) sa dá rozumieť tak, že úlohu nemusíme celú priradiť jednému procesoru,
ale rôzne časti môžme dať rôznym procesorom.
Keď  skúsime takto vyriešiť inštanciu z príkladu vyššie, na chvíľu by mohla skrsnúť nádej,
pretože optimálne riešenie je celočíselné. Stačí však, aby sme namiesto $T=10$ zobrali $T=7$ a optimálne 
priradenie bude $J_1\mapsto M_1$, $J_2\mapsto M_3$, $J_3\mapsto M_4$ a $J_4\mapsto M_2$ s cenou $21$,
pričom existuje relaxované riešenie

\begin{align*}
  x_{11} &=\frac{3}{7} &
  x_{22} &=\frac{49}{72} &
  x_{24} &=\frac{1}{8} & 
  x_{32} &=\frac{23}{72} &
  x_{34} &=0 &
  x_{41} &=\frac{4}{7} &
  x_{42} &=0 &
  x_{43} &=1 &
  x_{54} &=\frac{7}{8}
\end{align*}

\noindent
\begin{minipage}[t]{0.5\textwidth}
  \vskip 0pt
\noindent
s cenou $\frac{7019}{504}\approx 13.9$. Vidno aj, v čom je problém: napríklad $M_5$ je cenovo výhodná, ale
pomalá. V celočíselnom riešení ju nemôžeme použiť, ale v relaxovanom jej môžme prideliť takú porciu úlohy $J_4$,
aby sa presne vyplnil limit. Napriek tomu vidíme, že aj v tomto relaxovanom riešení sú niektoré hodnoty celočíselné
a na tom by sme chceli založiť našu stratégiu: zafixovať celočíselné hodnoty a zvyšok riešiť nejakým 
iným lineárnym programom. Zaujíma nás preto, či môže existovať inštancia, ktorá nemá v optimálnom relaxovanom
riešení žiadne celočíselné hodnoty.
Obrázok vpravo ukazuje, že môže. Zdá sa, že náš pôvodný plán využiť celočíselné hodnoty z relaxovaného riešenia
sa dostal do slepej uličky. Ako sa však ukáže, optimálne riešenia, ktoré nemajú 
\end{minipage}\begin{minipage}[t]{0.5\textwidth}
  \begin{myfiglabel}{0.7\textwidth}{svg/assignment2}{fig:mingap:ex}
    Optimálne priradenie pre limit $T=10$ je $M_2$ s cenou 9. 
    Optimum relaxovaného riešenia je $x_{11}=x_{21}=x_{31}=\frac{1}{3}$,
  s cenou 5.
  \end{myfiglabel}
\end{minipage}
\noindent
žiadne celočíselné hodnoty majú veľmi špeciálny tvar, a tak sa nám predsalen podarí situáciu zachrániť.
Pri návrhu iteratívneho algoritmu by sa neskôr ukázalo, že potrebujeme riešiť jemne všeobecnejšiu verziu 
relaxovaného programu. Ušetríme si preto robotu a uvedieme ju hneď od začiatku. Namiesto všetkých procesorov
budeme vyžadovať splnenie termínu iba od nejakej podmnožiny $M'\subseteq M$; navyše, každý procesor bude mať 
potenciálne iný termín. Dostávame teda program

\begin{equation}
\label{eq:mingap:lp}
\begin{array}{rrcll}
  {\rm minimalizovať}     & \multicolumn{1}{l}{ \sum\limits_{(i,j)\in E}x_{ij}c_{ij}}\\[4mm]
  {\rm pri\ obmedzeniach} & \sum\limits_{i\in M} x_{ij}& = &1 & \;\;\; \forall j\in J\\[4mm]
                          & \sum\limits_{j\in J} t_{ij}x_{ij}&\le & T_i & \;\;\; \forall i\in M'\\[4mm]
                        %  & \multicolumn{3}{c}{ x_{ij}\in\{0,1\}} & \;\;\; \forall i\in M, \; \forall j\in J
                          & x_{ij} & \ge & 0 & \;\;\; \forall i\in M, \; \forall j\in J
\end{array}
\end{equation}

\vskip 1ex
\noindent
Relaxácia programu (\ref{eq:mingap:ilp}) je špeciálnym prípadom (\ref{eq:mingap:lp}) pre $M'=M$ a $T_i=T$.
V ďalšom texte budeme symbolom $deg(k)$, pre $k\in J\cup M$  označovať stupeň procesora alebo úlohy
v grafe $G$ (t.j. počet hrán, ktoré z neho vychádzajú).
Pre celý algoritmus je kľúčová nasledovná charakterizácia optimálnych riešení:

\begin{lema}
  \label{lm:mingap:char}
  Nech \bm{x} je optimálne bázové riešenie programu (\ref{eq:mingap:lp}), kde pre všetky $i$, $j$ je
  $0<x_{ij}<1$. Potom existuje procesor $i\in M'$, pre ktorý buď $deg(i)\le 1$, alebo $deg(i)=2$ a 
  $\sum_{j\in J}x_{ij}\ge1$.
\end{lema}

\noindent
Lema~\ref{lm:mingap:char} hovorí, že ak sa po vyriešení programu (\ref{eq:mingap:lp}) ocitneme v situácii,
kde nie je žiadna celočíselná premenná, tak buď máme procesor, ktorý dokáže spracovávať
najviac jednu úlohu ($d(i)\le1$), alebo procesor, ktorý dokáže spracovávať práve dve úlohy a z každej z nich
spracováva kus (navyše, v súčte sú tie kusy $\ge1$).
Pri dôkaze lemy~\ref{lm:mingap:char} využijeme nasledujúce pomocné tvrdenie:

\begin{lema}
  \label{lm:mingap:tmp1}
  Nech \bm{x} je bázové riešenie programu (\ref{eq:mingap:lp}), v ktorom všetky $x_{ij}>0$. 
  Potom existuje množina
  $M''\subseteq M'$ veľkosti $|M''|=|E|-|J|$ taká, že pre všetky $i\in M''$ platí 
  $\sum\limits_{j\in J}t_{ij}x_{ij}=T_i$.
\end{lema}


\begin{dokaz}
  Pripomeňme si definíciu bázového riešenia (Definícia~\ref{dfn:LP:basis}). Najprv požadujeme program v normálnom 
  tvare $\max\limits_{\bm{x}}\{\bm{c}\tr\bm{x}\mid A\bm{x}=\bm{b},\;\bm{x}\ge0\}$, kde $A$ má plnú hodnosť
  (t.j. neobsahuje lineárne závislé riadky). Program (\ref{eq:mingap:lp}) upravíme tak, že zmeníme minimalizáciu
  na maximalizáciu a pre každé obmedzenie $\sum_{j\in J}t_{ij}x_{ij}\le T_i$ zavedieme rezervnú premennú
  $s_i$ a dostaneme \hbox{$\sum_{j\in J}t_{ij}x_{ij}+s_i= T_i$.} Získame tak maticu obmedzení 
  rozmerov $(n+m')\times (|E|+m')$
  v tvare
$$A=\left(\begin{array}{c|c}B&0\\\hline C&I\end{array}\right)$$
  kde $B\in\R^{n\times |E|}$ je matica obmedzení pre úlohy, $C\in\R^{m'\times|E|}$ je matica obmedzení pre procesory
  a $I$ je identická matica $m'\times m'$ zahŕňajúca rezervné premenné $s_i$.
  Čitateľ sa ľahko presvedčí, že riadky matice $A$ sú lineárne nezávislé
  (každý riadok v hornej časti obsahuje iné premenné).

  \centerline{\begin{minipage}[t]{0.9\textwidth}\vskip 0pt
      \mycaption{Napríklad pre zadanie (\ref{fig:mingap:ex}) dostávame program
        $$\max_{\bm{x}\in\R^6}\{\bm{c}\tr\bm{x}\mid A\bm{x}=\bm{b},\;\bm{x}\ge0\} $$
        kde
        \begin{align*}
          \bm{c}&=\cvect{-3\\-9\\-3\\0\\0\\0}&
          \bm{x}&=\cvect{x_{11}\\x_{21}\\x_{31}\\s_1\\s_2\\s_3}&
        A&=\left(\begin{array}{ccc|ccc}1&1&1&0&0&0\\\hline30&0 &0 &1&0&0\\0&10&0 &0 &1&0\\0&0&30&0&0&1 \end{array}\right)&
          \bm{b}&=\cvect{1\\T\\T\\T}
        \end{align*}
      }
  \end{minipage}}

  \noindent
  Podľa Definície~\ref{dfn:LP:basis}, báza má veľkosť $n+m'$ a všetky nebázové zložky musia byť nulové. Pretože
  predpokladáme, že všetky $x_{ij}>0$, v bázovom riešení musí byť $|E|+m'-(n+m')=|E|-n$ nulových hodnôt $s_i$.
  Každá nulová rezervná premenná $s_i$ dáva jeden procesor, pre ktorý platí $\sum_{j\in J}t_{ij}x_{ij}+s_i= T_i$.
\end{dokaz}


\begin{dokazpar}{Lemy~\ref{lm:mingap:char}}
  Majme bázové riešenie programu (\ref{eq:mingap:lp}), kde pre všetky $i,j$ je $0<x_{ij}<1$ a navyše pre
  všetky $i\in M'$ je $deg(i)\ge2$. 
  Pre každú úlohu $j\in J$ musí platiť 
  $deg(j)\ge2$: pretože  $\sum_{i\in M}x_{ij}=1$, keby $deg(j)=1$, musela by nejaká hodnota $x_{ij}=1$.
  Zoberme si množinu $M''$ z Lemy~\ref{lm:mingap:tmp1}. Platí
  $$|J|+|M''|=|E|=\frac{\sum\limits_{j\in J}deg(j) + \sum\limits_{i\in M}deg(i)}{2}
  \ge\frac{\sum\limits_{j\in J}deg(j) + \sum\limits_{i\in M'}deg(i)}{2}
  \stackrel{(\clubsuit)}{\ge}|J|+|M'|\ge |J| + |M''|
  $$

  \noindent
  Nerovnosť $(\clubsuit)$ platí preto, lebo pre $i\in M'$ je $deg(i)\ge 2$ z 
  predpokladu a $deg(j)\ge 2$ sme ukázali pred chvíľou pre všetky $j\in J$.
  Pretože prvý a posledný člen v sérii nerovností sa rovnajú, musia platiť rovnosti všade. 
  Pretože $M''\subseteq M'$, platí $M'=M''$. Zároveň, pretože všetky vrcholy z $J$ a $M'$ sú stupňa aspoň 2, 
  je $deg(j)=2$ pre všetky $j\in J$ a  $deg(i)=2$ pre všetky $i\in M'$
  jediná možnosť, ako uchovať v platnosti rovnosť v $(\clubsuit)$.
  Aby mohla platiť aj predchádzajúca rovnosť, musí byť
  $deg(i)=0$ pre všetky $i\in M\setminus M'$.

  \noindent
  Z toho ale vyplýva, že $G$ sa skladá z izolovaných vrcholov a vrcholov stupňa 2, čiže je tvorený
  dizjunktnými cyklami. Keďže $G$ je bipartitný, každý cyklus $C$ má párnu dĺžku, a preto obsahuje rovnaký počet
  $k$
  procesorov aj úloh. Keďže pre každú úlohu $j$ je $\sum_{i\in M'}x_{ij}=1$, je $\sum_{(i,j)\in C}x_{ij}=k$
  a nutne musí v $C$ existovať procesor $i$, pre ktorý je $\sum_{j\in J}x_{ij}\ge1$.
\end{dokazpar}

\noindent
S charakterizáciou bázových riešení pomocou Lemy~\ref{lm:mingap:char} sa nám začína črtať algoritmus riešenia.
Keď v riešení programu (\ref{eq:mingap:lp}) je nejaká premenná $x_{ij}=1$, priradíme úlohu $j$ procesoru $i$;
úloha $j$ tým bude vybavená a procesoru $i$ sa zmenší jeho termín $T_i$ (preto sme v programe (\ref{eq:mingap:lp})
chceli mať rôzne limity pre rôzne procesory). Ak je nejaká premenná $x_{ij}=0$, vyhodíme z grafu príslušnú
hranu, čím dostaneme menší graf (a teda menší problém). Ak je nejaký procesor $i\in M'$, pre ktorý $deg(i)=1$,
nemusíme pri ňom uvažovať žiaden termín: keďže je iba jedna úloha, ktorá mu potenciálne môže byť priradená
(a na začiatku sme bez ujmy na všeobecnosti predpokladali, že každá hrana má čas nanajvýš $T$), $i$ nikdy neprekročí 
termín. Posledná vec, čo sa môže po vyriešení programu (\ref{eq:mingap:lp}) stať, je že máme v $M'$ procesor $i$,
ktorý má stupeň 2 a $\sum_{j\in J}x_{ij}\ge1$. V tomto prípade tiež nemusíme uvažovať žiaden termín:
$i$ má stupeň 2, takže najväčšia možná hodnota $\sum_{j\in J}x_{ij}\le2$. Keďže v našom riešení je 
$\sum_{j\in J}x_{ij}\ge1$ a termín je stále splnený, aj bez akýchkoľvek obmedzení  termín nemôže byť prekročený 
viac ako o dvojnásobok. V oboch prípadoch teda dostaneme problém, ktorý je menší v tom, že sa zmenšila
množina $M'$. Poďme teraz túto tému rozviesť detailnejšie.
Algoritmus na riešenie problému \mingap bude vyzerať nasledovne
\newpage

\hrule
\begin{itemize}
  \item[1] $M':=M$, $\forall i:\;T_i:=T$, $F:=\emptyset$ 
    ($F$ je množina priradených hrán)
  \item[2] kým $J\not=\emptyset$
    \begin{itemize}
      \item[3\phantom{a}] nech \bm{x} je optimálne riešenie  programu (\ref{eq:mingap:lp}),\\
        vykonaj jednu z nasledovných možností:
      \item[4a] ak existuje $x_{ij}=0$,\\ vyhoď hranu $(i,j)$ z $G$
      \item[4b] ak existuje $x_{ij}=1$,\\
        $F:=F\cup\{(i,j)\}$, $J:=J\setminus\{j\}$, $T_i:=T_i-t_{ij}$
        \item[4c] inak nech $i\in M'$ také, že $deg(i)\le1$ alebo $deg(i)=2$ a $\sum_{j\in J}x_{ij}\ge1$\\
        $M':=M'\setminus\{i\}$
    \end{itemize}
\end{itemize}
\hrule
\vskip 3ex

\noindent
Každá iterácia cyklu na riadku 2 zmenší hodnotu $|E|+|J|+|M'|$, preto algoritmus po polynomiálnom počte
iterácií skončí. Ľahko vidno, že po skončení je každá úloha priradená nejakému procesoru. Ostáva nám ukázať,
že toto priradenie má optimálnu cenu a termín je prekročený najviac o dvojnásobok:

\begin{veta}
Uvedený algoritmus vyrieši v polynomiálnom čase problém \mingap. Nájdené riešenie má optimálnu cenu
a každý procesor skončí najneskôr v čase $2T$.
\end{veta}

\begin{dokaz}
  Už sme nahliadli, že algoritmus v polynomiálnom čase vráti nejaké riešenie. Teraz ukážeme, že toto riešenie
  má optimálnu cenu. Nech $c$ je cena optimálneho riešenia programu (\ref{eq:mingap:lp}) pre $M'=M$ a
  $T_i=T$. Zjavne $c\le OPT$.   
  Indukciou ukážeme, že v každej iterácii cena už priradených úloh $F$, spolu s 
  cenou riešenia aktuálneho programu (\ref{eq:mingap:lp}) z riadku 3 je dokopy nanajvýš $c$.
  V prvej iterácii tvrdenie platí triviálne. Ak sa vykoná riadok 4a, nezmení sa ani cena $F$, ani cena
  optima (\ref{eq:mingap:lp}). Ak sa vykoná riadok 4b, cena $F$ stúpne o $c_{ij}$ a cena optima 
  (\ref{eq:mingap:lp}) klesne aspoň
  o $c_{ij}x_{ij}=c_{ij}$. Ak sa vykoná riadok 4c, cena $F$ sa nezmení a cena optima
  (\ref{eq:mingap:lp}) nestúpne.


  \noindent
  Teraz ukážeme, že čas každého procesora je najviac $2T$.
  Označme $F_{i,\ell}$ čas, ktorý na procesore $i$ indukujú úlohy priradené do $F$ počas prvých
  $\ell-1$ iterácií, a podobne $T_{i,\ell}$ hodnotu $T_i$ na začiatku $\ell$-tej iterácie.
  Zafixujme si procesor $i$. Počas prvých iterácií, kým $i\in M'$, ostáva v platnosti
  $T_{i,\ell}+F_{i,\ell}\le T$. Uvažujme iteráciu $\ell$, v ktorej sa procesor $i$ vyhodí z $M'$ v riadku 4c.
  To sa môže stať dvoma spôsobmi:

  \noindent
  {\bfseries 1. $deg(i)\le1$}. Nech $j$ je jediná úloha, ktorá je spojená\footnote{Podľa správnosti by sme
    mali aj graf $G$ indexovať číslom iterácie $\ell$, lebo sa $G$ sa v priebehu výpočtu mení. Spoľahneme sa
  ale na to, že z kontextu je zjavné, o ktorý graf $G$ sa jedná.} v $\ell$-tej iterácii
  s $i$.
  Koľko času spotrebuje $i$ vo finálnom riešení? Čas, ktorý už má priradený, t.j. $T_{i,\ell}$, plus možno
  $t_{ij}$, ak sa mu $j$-ta úloha priradí. Viac spotrebovať nemôže, lebo žiadna iná úloha s ním už nie je spojená.
  Preto výsledný čas je nanajvýš $F_{i,\ell}+t_{ij}\le T_{i,\ell}+F_{i,\ell}+t_{ij}\le 2T$; posledná nerovnosť
  vyplýva z toho, že pre všetky hrany je $t_{ij}\le T$ a až po $\ell$-tú iteráciu bol $i\in M'$, takže
  $T_{i,\ell}+F_{i,\ell}\le T$.

  \noindent
  {\bfseries 2. $deg(i)=2$ a $\sum_{j\in J}x_{ij}\ge1$} Nech $j_1$, $j_2$ sú jediné dve úlohy, ktoré
  sú spojené s $i$. Opäť sa pýtame, koľko času môže spotrebovať $i$ vo výslednom riešení. Zjavne
  najviac $F_{i,\ell}+t_{ij_1}+t_{ij_2}$. Na začiatku $\ell$-tej iterácie platilo 
  $F_{i,\ell}+T_{i,\ell}\le T$. Pri riešení programu (\ref{eq:mingap:lp}) v $\ell$-tej iterácii
  je $T_{i,\ell}$ limit pre procesor $i$, preto ak \bm{x} je riešenie z riadku 3, platí
  $t_{ij_1}x_{ij_1}+t_{ij_2}x_{ij_2}\le T_{i,\ell}$.
  Preto platí $F_{i,\ell}\le T-t_{ij_1}x_{ij_1}+t_{ij_2}x_{ij_2}$ a odtiaľ dostaneme,
  že $i$ určite nespotrebuje viac času ako 
  
  \begin{eqnarray*}
    T-t_{ij_1}x_{ij_1}+t_{ij_2}x_{ij_2}+t_{ij_1}+t_{ij_2} &\le \\
    T + (1-x_{ij_1})t_{ij_1} + (1-x_{ij_2})t_{ij_2} &\le \\
    T + (1-x_{ij_1})T + (1-x_{ij_2})T &\le \\
    T (3-x_{ij_1}-x_{ij_2})&\le& 2T
  \end{eqnarray*}
  
  \noindent
  pričom posledná nerovnosť vyplýva z predpokladu, že $\sum_{j\in J}x_{ij}\ge1$.

\end{dokaz}
